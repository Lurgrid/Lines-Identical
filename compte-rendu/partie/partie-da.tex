\documentclass[12pt]{article}
\usepackage[french]{babel}
\usepackage[T1]{fontenc}
\usepackage{graphicx} % Required for inserting images
\usepackage[letterpaper,margin=3cm]{geometry}
\usepackage[export]{adjustbox}
\usepackage{listings}
\usepackage{tabularx}
\usepackage{multirow}
\usepackage[table]{xcolor}
\usepackage{svg}
\usepackage{tocloft}
\usepackage{algorithmic}
\usepackage{indentfirst}
\usepackage{hyperref}

\renewcommand{\arraystretch}{1.7}
\setlength{\arrayrulewidth}{1pt} % épaisseur de la ligne du 

\renewcommand{\thesection}{\Roman{section}} 
\renewcommand\thesubsection{\arabic{subsection}}

\setlength{\cftsecnumwidth}{2em} 
% ajuste la largeur de la colonne des numéros de section

\renewcommand{\algorithmicrequire}{\textbf{Pré-condition}}
            
\renewcommand{\algorithmicdo}{\textbf{faire}}
\renewcommand{\algorithmicend}{\textbf{Fin}}
\renewcommand{\algorithmicif}{\textbf{Si}}
\renewcommand{\algorithmicelse}{\textbf{Sinon}}
\renewcommand{\algorithmicthen}{\textbf{alors}}
\renewcommand{\algorithmicwhile}{\textbf{Tant que}}
\renewcommand{\algorithmicwhile}{\textbf{Pour chaque}}

\newcommand{\spacebox}{8pt}


\setlength{\parindent}{1cm}
\sloppy

\begin{document}
    \section{Module de tableau dynamique, da}

    \subsection{Présentation du module.}

    De nombreux problème dans ce projet réside dans l'ajout d'un nombre 
    quelqu'onque d'élement à une structure de donné. Notament dans la 
    lecture des lignes, le comptage des occurences de celle-ci ou leur 
    numérotations mais aussi dans la gestion de la liste de fichier à 
    traiter. C'est pour cela que nous avons opter pour la création d'un 
    module de gestion d'un objet proche des listes disponible dans d'autre 
    language tel que le python. 

    Afin d'être en accord avec l'implémentation des tableaux en C, la 
    spécification du module DA promet que les éléments stockés sont 
    contingue dans la mémoire et sans offset.
    
    \subsection{Pour et contre}

    Le point négatif d'une telle condition, est une perte de séparation 
    entre l'implémentation et la spécification du module. Or, elle permet 
    une simplification de l'uitilisation du module mais aussi des gains de 
    performance. On peut notament cité le traitement des lignes lus. 

    Une ligne lu est stoké dans un buffeur de type da. Sans la contrainte il
    nous est impossible de la comparer au chaine de charactère déjà lu 
    (les chaines déjà lu n'étant pas des da, car stocké des da ne 
    permetterait pas l'utilisation de fonction de comparaison comme strmcp 
    et strcoll). Il nous serait obligatoire d'allouer une nouvelle chaine 
    de la taille de la ligne présente dans le buffer puis d'y recopier la 
    ligne pour enfin pourvoir la comparer. Or de ce traitement on comprend 
    qu'il faudra allouer une chaine dans touts les cas, même si cette ligne
    est déjà lu ou même dans le cas ou le fichier qui est lu n'est pas le 
    premier (Les lignes lu sur des fichiers qui ne sont pas le premier ne 
    seront jamais sauvegrader). 
    
    Nous voillont bien que dans ces deux cas, cette restriction permet 
    d'économiser une allocation d'un doublon pour le permier et pour le 
    deuxième une d'allocation. Par exemple, pour deux fichier. Dont le 
    premier est composer que d'une ligne et le deuxième de n ligne. On 
    obtient les allocations suivantes:\\
    \begin{figure}[ht]
        \centering
        \begin{tabular}{|r|c|c|c|}
            \hline{}
                \cellcolor{gray!25}     & Premier fichier & Deuxième fichier & 
                Total d'allocation \\
            \hline{}
            Avec restriction  & 1 & 0 & 1 \\
            \hline{}
            Sans restriction & 1 & $n$ & n + 1\\
            \hline
        \end{tabular}
        \caption{Comparaison nombre d'allocation sans prendre en compte 
        celle du buffer}\label{tab-compar-da}
    \end{figure}

    Pour conclure, ce module n'a pas été d'une grande difficulter à 
    implémenter, n'y même à imaginer. La seul difficulter à bien sur été 
    d'accepter ou non la contrainte de continuéter que nous imposont à ce type.
\end{document}