\documentclass[12pt]{article}
\usepackage[french]{babel}
\usepackage[T1]{fontenc}
\usepackage{graphicx} % Required for inserting images
\usepackage[letterpaper,margin=3cm]{geometry}
\usepackage[export]{adjustbox}
\usepackage{listings}
\usepackage{tabularx}
\usepackage{multirow}
\usepackage[table]{xcolor}
\usepackage{svg}
\usepackage{tocloft}
\usepackage{algorithmic}
\usepackage{indentfirst}
\usepackage{hyperref}

\renewcommand{\arraystretch}{1.7}
\setlength{\arrayrulewidth}{1pt} % épaisseur de la ligne du 

\renewcommand{\thesection}{\Roman{section}} 
\renewcommand\thesubsection{\arabic{subsection}}
\setlength{\cftsecnumwidth}{2em} % ajuste la largeur de la colonne des numéros de section
\renewcommand{\algorithmicrequire}{\textbf{Pré-condition}}
            
\renewcommand{\algorithmicdo}{\textbf{faire}}
\renewcommand{\algorithmicend}{\textbf{Fin}}
\renewcommand{\algorithmicif}{\textbf{Si}}
\renewcommand{\algorithmicelse}{\textbf{Sinon}}
\renewcommand{\algorithmicthen}{\textbf{alors}}
\renewcommand{\algorithmicwhile}{\textbf{Tant que}}
\renewcommand{\algorithmicwhile}{\textbf{Pour chaque}}

\newcommand{\spacebox}{8pt}


\setlength{\parindent}{1cm}
\sloppy

\begin{document}
    \section{Module de gestion d'option, optl}
    Le dévellopement du module d'option est la partie du projet qui nous a été 
    le plus chronofage. En effet, nous avons dès le début eu pour objectifs de 
    produire un module générique, réutilisable pour de nombreux programme. A 
    l'instar du module getopt qui nous sembler manquer certain fonctionnalités
    très importante tel que la représentation des options par une chaine de 
    charactère (représentation longue), mais aussi d'autre chose présente dans 
    les option linux. 

    Dans un premier temps, il nous a fallu effectue des recherches sur la 
    gestion d'option des programmes linux. C'est après ces recherches que nous 
    avons remarquer avec stupeur que les programme linux n'ont pas de norme pour 
    définir pour leur option. Nous avons donc décider de nous en créer une en 
    nous inspirant des commandes tel que cat, ls ou encore rm du système linux.

    Les options peuvent être représenter par deux indentificateur, une version 
    courte et une longue. Les deux ne sont pas obligatoire mais une des deux 
    doit au moins exister. Toute option à possibliter d'intérompre le traitement 
    d'option. Toute option peut exiger un arguement pour effectué son 
    traitement. Toutes les options peuvent avoir une description. 

    \subsection{Défintion d'option}

    \subsubsection{Les options courte}

    Une option courte est composer d'un `-' suivit d'un charactère 
    alphanumérique. Pour donner un paramettre à une option courte il s'uffit de 
    le séparer d'un espace On peut faire appelle à plusieur option courte en un 
    seul  appelle, en suivant le '-` des charactère représentant les options 
    voulus. Cependant, il ne peut y avoir dans cette forme d'appelle qu'un seul 
    paramettre demandant un arguement et il doit alors être le dernier de la 
    list. 

    \subsubsection{Les options longue}

    Une option longue est préfixé par la chaine `--'. Pour donner un paramettre 
    à une longe il faut s'éparer celle-ci de son argument par le charactère `='. 
    <parler de l'autocomplétion des options longues>

    \subsubsection{Option help}

    Nous avons aussi décider de rendre obligatoire une option, l'option `help'.
    Cette option représenter par `-h', `--help', doit afficher une possible 
    description du programme, comment l'utiliser mais aussi la liste de toutes 
    les options suivit de leur possible description. Cette option intéromp le 
    traitement des possibles option suivante. 

    \subsection{Implémentation}

    \subsubsection{Spécification}

    Pour l'implémentation nous avons décider de permettre à l'utilisateur de 
    pouvoir modifier certain chose. Notament les préfixe des option courte et 
    longue (On peut notament envisager des utilisateur originaire windows qui 
    préférerait utiliser le `\\' au `-'), l'indentificateur des deux 
    représentation de l'option `help'. Nous avons aussi été contraint d'ajouter 
    un spécifieur permétant de garantir que la valeur suivant celui-ci n'est pas
    une option (lui aussi est modifiable par l'utilisateur du module). Sans ce 
    spécifieur, l'utilisateur ne pourrait pas rentrer la valeur `--help' en 
    prenant cette valeur non pas comme une option mais une valeur à traiter. Ce 
    spécifieur vaux par défault la chaine `--'. De plus, nous avons mis en place 
    un possible traitement sur les éléments qui ne sont pas des options.
    
    \subsubsection{Le code}

    Pour regrouper toutes les informations nécessaire à la gestion d'une option 
    nous avons donc mis en place un type optparam regroupant toutes ces 
    informations. La fonction opt\_init initialise une instance de ce type. La 
    véritable difficulter à été dans la conception de la fonction de traitement,
    opt\_parse. Cette fonctoin peut être diviser de la façon suivante:

    \newpage

    \begin{figure}[ht]
        \begin{algorithmic}
            \WHILE{argument \textbf{de} tableau-argument}
                \IF{argument représente NEXT\_NOPT}
                    \STATE{
                        \vspace{\spacebox}
                        \fcolorbox{red}{white}{
                        \begin{minipage}[c][8ex]{13cm}
                            Traitement de NEXT\_NOPT (le spécifieur que la 
                            prochaine valeur ne doit pas être considérer comme 
                            une option). 
                        \end{minipage}
                        }
                        \vspace{\spacebox}
                        }
                \ELSIF{argument est une option longue}
                    \STATE{
                        \vspace{\spacebox}
                        \fcolorbox{green}{white}{
                        \begin{minipage}[c][8ex]{13cm}
                            Traitement de l'option longue. Pour cela, un appelle
                            a la fonction opt\_parse\_long visent a trouver 
                            l'option dont argument est le seul préfix. Puis la 
                            fonction liée à l'option est éxecuter. 
                        \end{minipage}
                        }
                        \vspace{\spacebox}
                        }
                \ELSIF{argument est une option courte}
                    \STATE{
                        \vspace{\spacebox}
                        \fcolorbox{black}{white}{
                        \begin{minipage}[c][8ex]{13cm}
                            Traitement des possibles options courtes. Pour ce 
                            faire, un traitement sur chaque charactère est 
                            effectué permettant de trouver toutes les 
                            options représenté dans argument. Puis effectue le 
                            traitement liés à ces appelles.
                        \end{minipage}
                        }
                        \vspace{\spacebox}
                        }
                \ELSE{}
                    \STATE{
                        \vspace{\spacebox}
                        \fcolorbox{blue}{white}{
                        \begin{minipage}[c][8ex]{13cm}
                            Traitement de ce qui n'est pas une option, à l'aide 
                            la fonction hdl\_dlt possiblement fournit par 
                            l'utilisateur.
                        \end{minipage}
                        }
                        \vspace{\spacebox}
                        }
                \ENDIF{}
            \ENDWHILE{}
        \end{algorithmic}
        \caption{Traitement du tableau des arguement par la fonction 
        opt\_process.}
    \end{figure}

    \subsection{Pour et contre du module}

    Ce module à été consu pour être utiliser dans de nombreux cas. Il pourrait 
    donc reservi pour des futurs projets. Cette généraliter entraine par contre 
    une grande difficulter quand à la compréhension de ces fonctionnalités, on 
    peut notament citer la Spécification de la fonction otp\_process qui est 
    trop longue.
    Un point fort du reste qu'il est complet, il met en pratique toute les 
    fonctionnalités des commandes linux les plus connu (rm, ls, cat), notament d
    des fonctionnalités tel que `l'autocomplétion' des options longues, la prise
    en charge des multiple option courte ou encore le spécifieur NEXT\_NOPT 
    (next is not an option). Un autre point fort, la compléxiter du traitement 
    des options, toues les options ne sont parcourue qu'au plus deux fois pour 
    exercuter leur traitement (avec la possible comparaison avec l'option 
    `help'). De même ce module ne nécessite que le stockage des objets de type 
    optparam pour traiter les options.

    \section{Bibli}
    Lien: \\
    Présentation des divers `tradition' de gestion d'option pour unix et gnu, 
    \url{http://www.catb.org/~esr/writings/taoup/html/ch10s05.html}.\\
    Descrpition de la gestion des option en ligne à la GNU, 
    \url{https://www.gnu.org/software/gawk/manual/html_node/Options.html}.\\
    Documentation du module getopt, pour possiblement faire une comparaison 
    entre ce module et optl, 
    \url{https://www.gnu.org/software/libc/manual/html_node/Getopt.html}.
\end{document}