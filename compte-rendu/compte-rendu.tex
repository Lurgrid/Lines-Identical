\documentclass[12pt]{article}
\usepackage[french]{babel}
\usepackage[T1]{fontenc}
\usepackage{graphicx} % Required for inserting images
\usepackage[letterpaper, margin=3cm]{geometry}
\usepackage[export]{adjustbox}
\usepackage{listings}
\usepackage{tabularx}
\usepackage{multirow}
\usepackage[table]{xcolor}
\usepackage{tocloft}
\usepackage{algorithmic}
\usepackage{indentfirst}
\usepackage{hyperref}
\usepackage{array}
\usepackage{float}
\usepackage{caption}
\usepackage{tcolorbox,url}
\usepackage{subfig}
\usepackage{fancyvrb}
\usepackage{tikz}
\usetikzlibrary{positioning}
\tcbuselibrary{skins,xparse,listings}

\newtcblisting{ubuntu}{colback=violet!50!black,
colupper=white,colframe=gray!65!black,listing only,
listing options={style=tcblatex,language=sh,escapeinside=``,},
title={\textcolor{orange}{\Huge{$\bullet$}}{\textcolor{gray}
{\Huge{$\bullet\bullet$}}}},
every listing line={\MyUbuntuPrompt}}
\pgfkeys{/ubuntu/.cd,
user/.code={\gdef\MyUbuntuUser{#1}},user={},
host/.code={\gdef\MyUbuntuHost{#1}},host={},
color/.code={\gdef\MyUbuntuColor{#1}},color=white,
prompt char/.code={\gdef\MyUbuntuPromptChar{#1}},prompt char=\#,
root/.style={user=root,host=ubuntu,color=lime,prompt char=\#},
bob/.style={user=bob,host=remotehost,color=cyan},
}
\newcommand{\SU}[1]{\pgfkeys{/ubuntu/.cd,#1}

\gdef\MyUbuntuPrompt{\textcolor{\MyUbuntuColor}{\small\ttfamily\bfseries
\MyUbuntuUser@\MyUbuntuHost{\textcolor{white}:}\textcolor{cyan!60}
{$\scriptstyle\sim$}{\textcolor{white}\MyUbuntuPromptChar \hphantom{p}}}}}
\newcommand{\StartConsole}{\gdef\MyUbuntuPrompt{}}

\definecolor{mGreen}{rgb}{0,0.6,0}
\definecolor{mGray}{rgb}{0.5,0.5,0.5}
\definecolor{mPurple}{rgb}{0.58,0,0.82}
\definecolor{backgroundColour}{rgb}{0.95,0.95,0.94}

\lstdefinestyle{CStyle}{
    backgroundcolor=\color{backgroundColour},   
    commentstyle=\color{mGreen},
    keywordstyle=\color{mPurple},
    numberstyle=\tiny\color{mGray},
    stringstyle=\color{mGreen},
    basicstyle=\footnotesize,
    breakatwhitespace=false,         
    breaklines=true,                 
    captionpos=b,                    
    keepspaces=true,                 
    numbers=left,                    
    numbersep=5pt,                  
    showspaces=false,                
    showstringspaces=false,
    showtabs=false,                  
    tabsize=2,
    language=C
}

\newcommand{\textLabel}{}
\newcommand{\settextLabel}[1]{%
  \renewcommand{\textLabel}{#1}
}

\DefineVerbatimEnvironment{textfile}{Verbatim} {
    fontsize=\footnotesize,
    %
    frame=lines,  % top and bottom rule only
    framesep=2em, % separation between frame and text
    rulecolor=\color{gray},
    %
    label=\fbox{\color{black}\textLabel},
    labelposition=topline,
    numbers=left
}

\newenvironment{myverbatim}[1]
{\verbatim\fontfamily{cmtt}\fontsize{10pt}{12pt}\selectfont
 \textbf{#1}\newline}
{\endverbatim}

\renewcommand{\arraystretch}{1.7}
\setlength{\arrayrulewidth}{1pt} % épaisseur de la ligne du 

\renewcommand{\thesection}{\Roman{section}} 
\renewcommand\thesubsection{\arabic{subsection}}

\setlength{\cftsecnumwidth}{3em} 
% ajuste la largeur de la colonne des numéros de section

\renewcommand{\algorithmicrequire}{\textbf{Pré-condition}}
            
\renewcommand{\algorithmicdo}{\textbf{faire}}
\renewcommand{\algorithmicend}{\textbf{Fin}}
\renewcommand{\algorithmicif}{\textbf{Si}}
\renewcommand{\algorithmicelse}{\textbf{Sinon}}
\renewcommand{\algorithmicthen}{\textbf{alors}}
\renewcommand{\algorithmicwhile}{\textbf{Tant que}}
\renewcommand{\algorithmicwhile}{\textbf{Pour chaque}}

\newcommand{\spacebox}{8pt}

\definecolor{blue-s}{HTML}{9CA3DB}
\definecolor{blue-t}{HTML}{5c8ac4} 
\definecolor{blue-u}{HTML}{7AA9FF} 

\setlength{\parindent}{1cm}
\sloppy

%--- begin document ------------------------------------------------------------

\title{Projet d'Algorithmique}
\author{Edouard.H Théo.R.V}
\date{2022--2023}

\begin{document}

    \begin{figure}
        \includegraphics[scale=0.3, right]{logo-univ-rouen-normandie-noir.png}
    \end{figure}
    
    \maketitle

    \begin{abstract}
        Ce document constitue notre compte rendu du projet d'algorithmique 
        `Lines Identical (lnid)'.
        
        Dans un premier temps, nous présenterons une 
        brève description du projet. Ensuite, nous expliquerons comment nous 
        l'avons construit. Nous passerons ensuite en revue les différents 
        modules utilisés en détail. Enfin, nous présenterons une mise en 
        pratique avec des tests intéressants ainsi que des tests de performance.
        Pour conclure, nous aborderons les diverses limitations potentielles de 
        cette implémentation, ainsi que les améliorations possibles. Nous ferons
         également une rétrospective de la production de ce projet.
    \end{abstract}

    \newpage

    \tableofcontents

    \newpage

    \begin{figure}[H]  
         
        \centering
        \begin{tabular}{|c|c|c|}
            \hline
                \cellcolor{blue-s!25} -h & \cellcolor{blue-s!25} --help  &
                \cellcolor{blue-t!70}\\
            \hline
                \multicolumn{3}{|c|}{\cellcolor{blue-u!7}Affiche sur la sortie 
                standard la documentation de l'exécutable}\\
            \hline
                \cellcolor{blue-s!25} -u  & \cellcolor{blue-s!25} --uppercasing 
                & \cellcolor{blue-t!70} \\
            \hline
                \multicolumn{3}{|c|}{\cellcolor{blue-u!7}Tous les caractères 
                traiter seront en sortie en majuscule.}\\
            \hline 
                \cellcolor{blue-s!25} -N & \cellcolor{blue-s!25} --no-color  & 
                \cellcolor{blue-t!70} \\
            \hline
                \multicolumn{3}{|c|}{\cellcolor{blue-u!7}Enlève les couleurs de 
                l'affichage du programme.}\\
            \hline
                \cellcolor{blue-s!25} -a & \cellcolor{blue-s!25} --avl & 
                \cellcolor{blue-t!70} \\
            \hline
                \multicolumn{3}{|c|}{\cellcolor{blue-u!7}Utilise les avls pour 
                la gestion des lignes.}\\
            \hline
                \cellcolor{blue-s!25} -V & \cellcolor{blue-s!25} --version & 
                \cellcolor{blue-t!70} \\
            \hline
                \multicolumn{3}{|c|}{\cellcolor{blue-u!7}Affiche la version du  
                logiciel.}\\
            \hline
                \cellcolor{blue-s!25} -f CLASS & \cellcolor{blue-s!25} 
                --filter=CLASS & \rule{0pt}{2.6em}\cellcolor{blue-s!25}
                {\parbox{9cm}{ \cellcolor{blue-t!70}Avec \texttt{CLASS} l’un des 
                suffixes\ldots des douze tests\\ d’appartenance à une catégorie 
                de caractères is\ldots de l’en-tête standard $<$ctype.ht$>$}} \\
                [1.3em]
            \hline
                \multicolumn{3}{|c|}{\cellcolor{blue-u!7}\rule{0pt}{2em}
                \parbox{15cm}{Ne prend en compte au traitement que les 
                Caractères répondant au test  de présence dans l'ensemble 
                CLASS, fonction de test étant is.CLASS.}}\\[.8em]
            \hline
                \cellcolor{blue-s!25} -s WORD & \cellcolor{blue-s!25} 
                --sort=WORD & \cellcolor{blue-t!70}Avec \texttt{WORD} valant
                 soit \texttt{standard} soit \texttt{locale} \\
            \hline
                \multicolumn{3}{|c|}{\cellcolor{blue-u!7}\rule{0pt}{2em}
                \parbox{15cm}{{Trie les valeurs sur la sortie en prenant en 
                compte l'ordre \texttt{WORD} avec \texttt{standard}} 
                correspondant à l'ordre du  `C' et \texttt{locale} celui du 
                système de l'utilisateur.}}\\[1em]
            \hline
        \end{tabular}
        \captionsetup{position=bottom}
        \caption{Tableau des options}\label{table-opt}
    \end{figure}

    \newpage

    \section{Lines Identical, lnid}

    \subsection{Descrpition du programmes}

    Lnid, programme (écrit en C) élaborant le traitement suivant :
    \begin{enumerate}
        \item[] Si un seul fichier lui est fourni, alors affiche la liste 
        des numéros des lignes redondantes puis affiche ces lignes.
        \item[] Sinon si, plusieurs fichiers lui sont fournis, affiche pour
         tous les fichiers le nombre d'occurrences de ligne commune et 
         affiche ces lignes.
        \item[] Sinon si aucun fichier n'est fournie alors effectue le 
        traitement de fichier unique sur les lignes présente sur l'entrée 
        standard.
    \end{enumerate}

    Ils visent à être dans la lignée des commandes Linux tel que \textbf{cat} ou 
    \textbf{ls}. Il implémente donc des fonctionnalités telles que le caractère 
    \textbf{'-'} qui doit être considéré comme identificateur de l'entrée 
    standard ou encore \textbf{'- -'} qui prévient que l'argument qui le suit 
    doit être considéré comme un fichier. De plus, diverses options sont 
    disponible ajoutant des fonctionnalités. Voir ci-dessus pour leurs 
    spécifications, figure~\ref{table-opt}.

    \subsection{Implémentation du programme}
    Il y a déjà deux cas a dissocié, celui du fichier seul fichier et le cas de 
    plusieurs fichiers. En réalité, nous pouvons les regrouper en deux 
    cas, le cas du premier fichier et le cas des autres fichiers.

    Pour la lecture de ligne, ne consomme pas trop de mémoire à l'utilisateur
    (par l'allocation d'un buffer a chaque ligne lue), nous avons donc décidé de 
    mettre en place un buffer de type \textbf{da} (voir section~\ref{test} pour
    analyse complète de ce choix). Pour cette lecture, nous utilisons donc la
    fonction \textbf{fnlines} qui ajoute donc au buffer chaque lettre lus (si 
    l'utilisateur a choisi une option filtrante ou transformante des caractères 
    alors le caractère sera d'abord transformer puis filtrer et enfin ajouté) 
    sur le fichier.

    Chaque ligne lue ne sera pas traiter de la même façon selon les deux cas 
    évoquer précédemment :

    \begin{enumerate}
        \item[] Pour le premier cas toutes les lignes lus seront tester pour 
        voir si elles ont d'ores et déjà était lu. Si c'est le cas alors pour le 
        cas d'entrée d'un fichier unique ou de plusieurs, nous avons 
        respectivement l'ajoute du numéro de ligne dans la liste des lignes ou 
        de l'incrémentation de son nombre d'occurrences. Sinon associe cette ligne 
        a cette ligne un compteur initialement mis à 1 ou une liste de ligne 
        avec comme première valeur le numéro de cette ligne respectivement selon 
        qu'il y a plusieurs fichiers ou un fichier
        \item[] Pour les autres fichiers les lignes seront tester pour voir si 
        elle, on dort et déjà était lu. Si c'est le cas alors incrémente le 
        compteur associé à cette ligne. Sinon cette ligne n'est pas traitée
    \end{enumerate}

    Pour associer une ligne à son compteur/liste de numéro de lignes, vous 
    proposons deux types de structure par défaut une table de hashage et avec 
    une option un AVL (voir section~\ref{test} pour un comparatif de ces deux 
    solutions). Dans les deux les données présente dans ces structures sont 
    stocker dans un \textit{fourre-tout} (holdall) (voir section~\ref{holdall} 
    pour plus d'information sur l'utilisation de celui-ci).

    Une fois ce traitement effectuait, nous affichons donc le résultat du 
    traitement escompte par l'intermédiaire de l'affiche des diverses valeurs 
    présente dans le \textit{fourre-tout} de ligne. Cet affichage peut se 
    décrire de la manière suivante sellons les deux cas cités au début de cette 
    sous-section.

    \begin{figure}[H]
        \centering
        \begin{minipage}[b]{0.4\linewidth}
            \settextLabel{file1.txt}
            \begin{textfile}
Tester un programme démontre la
présence de 
bugs
pas leur absence.
bugs
présence de 
Tester un programme démontre la
            \end{textfile}
        \end{minipage}
        \hspace{0.1\linewidth}
        \begin{minipage}[b]{0.4\linewidth}
            \settextLabel{file2.txt}
            \begin{textfile}
La véritable valeur des tests 
n'est pas qu'ils détectent des 
bugs
dans le code, mais qu'ils 
détectent des insuffisances 
dans les méthodes, la 
concentration et les compétences 
de ceux qui conçoivent et 
produisent le code.
            \end{textfile}
        \end{minipage}   
        \vspace{10pt}  
        \SU{user=root,host=ubuntu,color=lime}
        \begin{ubuntu}
./lnid file1.txt `\StartConsole`
file1.txt
1,7     Tester un programme d`é`montre la
2,6     pr`é`sence de
3,5     bugs
`\SU{user=root,host=ubuntu,color=lime}`
./lnid file1.txt file2.txt `\StartConsole`
file1.txt file2.txt
2       1       bugs
        \end{ubuntu}
        \caption{\centering Affichage du programme pour un et deux fichiers. Les 
        flèches représente des tabulations.}
    \end{figure}

    L'implémentation de l'option de tri (\textbf{-s | --sort}) par un trie du 
    \textit{fourre-tout} (à l'aide d'un trie fusion).

    Tout au long du programme chaque opération peuvent entrainer un échec se 
    voit correspondre une gestion d'erreur lui étant associé. On peut notamment 
    les erreurs d'allocation dynamique, les erreurs de lecture/ouverture des 
    fichiers ou encore des erreurs quand a la gestion des options entré par 
    l'utilisateur. Toutes ces erreurs donne lieu à des messages envoyé sur la 
    sortie d'erreur (en couleur selon le choix de l'utilisateur avec l'option 
    \textbf{--no-color}).

    \newpage

    \section{Gestion des objets dynamique, holdall}\label{holdall}

    \subsection{Explication de l'objectif du module}
    
    De nombreux problèmes peuvent être résolus à l'aide de structure de donnée. 
    Pour faciliter leur implémentation en C, une façon commune est de passer par 
    des structures allouées dynamiquement. Or certaines de ces structures 
    demandent un nombre important de ces allocations dynamiques. On peut notamment
    cité les table de hachage, dont il est courant pour chaque valeur d'être 
    stoker dans un maillon lui-même gérer par une liste allouée dynamiquement 
    (voir schéma d'implémentation du module hastable figure \ref{hastable-fig}). 
    De plus, certaine de ces implémentations de structure peuvent engager des 
    contraintes, telles que la suppression d'une de ces allocations pour un certain
    traitement. On peut notamment citer des implémentations d'ensemble à l'aide de 
    liste. De telle structure n'aillant pas de requête de parkour, il faut alors 
    vidé toutes les valeurs de cet ensemble pour les affichés par exemple. 
    C'est dans ce cas que toutes les allocations devraient être désallouées et 
    alors, il  faudrait tous re-ajouter pour faire de nouvelles opérations sur 
    cet ensemble. Alors que si ces données étaient stockées dans une autre 
    structure, on pourrait plus ou moins les parcourir simplement selon cette 
    structure. On voit bien que dans notre implémentation l'utilisation d'une 
    liste simplement chainé est tous a fait justifier.

    \subsection{Mise en place dans ce projet}

    Dans notre projet, de nombreuses donné sont alloués dynamiquement. Les 
    lignes lues, les numéros de ces lignes. C'est pour cela que l'utilisation du 
    `fourre-tout' est tous a fait justifier. De ce fait, les structures 
    provenant du module da et avl n'ont absolument pas besoin d'avoir des
    opérations telles que la libération de mémoire pour les valeurs qu'elles 
    contiennent. De ce fait, l'utilisation du `four-tout' permet la 
    simplification du développement de certain module (par les gestions des 
    donné contenu dans les structures qui reviennent à l'utilisateur), mais 
    aussi une certaine lisibilité dans le code (On voit très bien dans lnid.c,
    que les lignes et leurs compteurs sont stockés tous deux dans des fourre-tout).

    \newpage

    \section{Module de tableau dynamique, da}

    \subsection{Présentation du module.}

    De nombreux problèmes dans ce projet réside dans l'ajout d'un nombre 
    quelconque d'élément à une structure de donnée. Notamment dans la 
    lecture des lignes, le comptage des occurrences de celle-ci ou leur 
    numérotations, mais aussi dans la gestion de la liste de fichier à 
    traiter. C'est pour cela que nous avons opté pour la création d'un 
    module de gestion d'un objet proche des listes disponibles dans d'autre 
    langage tel que le python. 

    Afin d'être en accord avec l'implémentation des tableaux en C, la 
    spécification du module DA promet que les éléments stockés sont 
    contigüe dans la mémoire et sans offset.
    
    \subsection{Avantage, inconvénient}

    Le point négatif d'une telle condition est une perte de séparation 
    entre l'implémentation et la spécification du module. Or, elle permet 
    une simplification de l'utilisation du module, mais aussi, des gains de 
    performance. On peut notamment citer le traitement des lignes lues. 

    Une ligne lue est stockée dans un buffer de type da. Sans cette contrainte il
    nous est impossible de la comparer aux chaines de caractère déjà lues 
    (les chaines déjà lues n'étant pas des da, car stocké des da ne 
    permettrait pas l'utilisation de fonction de comparaison comme strmcp 
    et strcoll). Il nous serait obligatoire d'allouer une nouvelle chaine 
    de la taille de la ligne présente dans le buffer puis d'y recopier la 
    ligne pour enfin pourvoir la comparer. Or de ce traitement, on comprend 
    qu'il faudra allouer une chaine dans tous les cas, même si cette ligne
    est déjà lu ou même dans le cas ou le fichier qui est lu n'est pas le 
    premier (Les lignes lues sur des fichiers qui ne sont pas le premier ne 
    seront jamais sauvegarder). 
    
    Nous voyons bien que dans ces deux cas cette restriction permet 
    d'économiser une allocation d'un doublon pour le premier et pour le 
    deuxième une d'allocation. Par exemple, pour deux fichiers dont le 
    premier est composé que d'une ligne et le deuxième de n ligne, on 
    obtient les allocations suivantes :\\
    \begin{figure}[ht]
        \centering
        \begin{tabular}{|r|c|c|c|}
            \hline{}
                \cellcolor{gray!25}     & Premier fichier & Deuxième fichier & 
                Total d'allocation \\
            \hline{}
            Avec restriction  & 1 & 0 & 1 \\
            \hline{}
            Sans restriction & 1 & $n$ & n + 1\\
            \hline
        \end{tabular}
        \caption{Comparaison nombre d'allocations sans prendre en compte 
        celle du buffer}\label{tab-compar-da}
    \end{figure}

    Pour conclure, ce module n'a pas été d'une grande difficulté à 
    implémenter n'y même à imaginer. La seule difficulté a été  
    d'accepter ou non la contrainte de continuité que nous imposons à ce type.

    \section{Hastable / AVL}

    La liaison entre une ligne et son nombre d'occurrences ou son numéro 
    d'apparition, peut être résolue par l'utilisation de structure de donnée qui 
    visent à joindre ces informations, pour pouvoir les récupérées. Nous avons 
    décidé de mettre deux types de ces structures à disposition, une d'arbres 
    binaire et l'autre de table de hachage. 

    \subsection{Hastable}

    Constituant l'implémentation d'une table de hachage, cette structure permet 
    à partir d'une clé, une valeur. Dans notre cas, la clé est ligne déjà lu et 
    la valeur ces numéros de lignes (dans le cas d'un seul fichier) ou son 
    nombre d'occurrences (dans le cas de plusieurs fichiers). Cette structure est 
    choisie pour répondre à ce problème. En effet, théoriquement une 
    table de hachage permet un accès presque constant à une valeur par sa clé. 
    Malgré tout, en pratique cet accès dépends exclusivement de la fonction 
    de hachage utilisé, c'est pour cela que dans la section de test, nous mettons 
    à disposition divers exemples d'exécutions avec des fonctions de hachages 
    différentes. De plus, en pratique, l'ajout dans une table de hachage peut
    entraîner un agrandissement de celle-ci et donc, recalculé de toutes les 
    valeurs en sont sein. L'implémentation de cette table est représenté dans 
    la figure : 
    
    \begin{figure}[H]
        \centering
        \begin{tikzpicture}
            \node[draw, rectangle, minimum size=1cm, color=black!100] (p) {};
            \node[draw, rectangle, below=of p, minimum width=2.5cm, minimum height=5cm] (struct) {};
            \draw[->, ultra thick, line width=1.5pt] (p.mid) -- (struct.north);
            \draw[fill=black] (p.mid) ++(0,-0.1cm) circle (0.07);

            \node[draw, minimum size=1cm, below left=-0.75cm of struct.center] (child1) {2};
            \node[draw, minimum size=1cm, above=-1.5cm of struct.north] (child2) {};
            \node[draw, minimum size=1cm, below=-1.5cm of struct.south] (child3) {2};

            \node[above, yshift=-0.1cm] at (child3.north) {lbnslots};
            \node[above, yshift=-0.1cm] at (child2.north) {hasharray};
            \node[above, yshift=-0.1cm] at (child1.north) {nfreeentries};

            \node[draw, rectangle, draw=none, minimum width=1cm, minimum height=4cm, xshift=3cm, below=of p] (chain) {};

            \draw[->, ultra thick, line width=1.5pt] (child2.mid) ++(0,-0.1cm) .. controls +(up:28mm) and +(up:14mm) .. (chain.north);
            \draw[fill=black] (child2.mid) ++(0,-0.1cm) circle (0.07);

            \node[draw, minimum size=1cm, above=-1cm of chain.north] (node1) {};
            \node[draw, minimum size=1cm, below=0cm of node1.south, yshift=0.5pt] (node2) {};
            \node[draw, minimum size=1cm, below=0cm of node2.south, yshift=0.5pt] (node3) {};
            \node[draw, minimum size=1cm, below=0cm of node3.south, yshift=0.5pt] (node4) {};

            \draw[fill=black] (node1.mid) ++(0,-0.1cm) circle (0.07);
        
            \node[draw, minimum height=1.75cm, minimum width=4cm, right= of node1] (next11) {};
            \node[draw, minimum height=1.75cm, minimum width=4cm, right= of next11] (next12) {};

            \node[draw, minimum size=0.75cm, right=1.4375cm of node1] (key11) {};
            \node[above, yshift=-0.1cm] at (key11.north) {key};

            \node[draw, minimum size=0.75cm, right=0.4375cm of key11] (key12) {};
            \node[above, yshift=-0.1cm] at (key12.north) {value};

            \node[draw, minimum size=0.75cm, right=0.4375 of key12] (key13) {};
            \node[above, yshift=-0.1cm] at (key13.north) {next};

            \draw[->, ultra thick, line width=1.5pt] (key13.mid) ++(0,-0.1cm) -- (next12);
            \draw[fill=black] (key13.mid) ++(0,-0.1cm) circle (0.07);

            \node[draw, minimum size=0.75cm, right=-3.5625cm of next12] (key21) {};
            \node[above, yshift=-0.1cm] at (key21.north) {key};

            \node[draw, minimum size=0.75cm, right=0.4375cm of key21] (key22) {};
            \node[above, yshift=-0.1cm] at (key22.north) {value};

            \node[draw, minimum size=0.75cm, right=0.4375cm of key22] (key23) {};
            \node[above, yshift=-0.1cm] at (key23.north) {next};

            \draw (key23.south west) -- (key23.north east);
            \draw (key23.north west) -- (key23.south east);
            
            \draw (node2.south west) -- (node2.north east);
            \draw (node2.north west) -- (node2.south east);

            \node[draw, minimum height=1.75cm, minimum width=4cm, right= of node3] (next31) {};

            \draw (node4.south west) -- (node4.north east);
            \draw (node4.north west) -- (node4.south east);

            \draw[fill=black] (node3.mid) ++(0,-0.1cm) circle (0.07);

            \draw[->, ultra thick, line width=1.5pt] (node1.mid) ++(0,-0.1cm) -- (next11);
            \draw[->, ultra thick, line width=1.5pt] (node3.mid) ++(0,-0.1cm) -- (next31);

            \node[draw, minimum size=0.75cm, right=1.4375cm of node3] (key31) {};
            \node[above, yshift=-0.1cm] at (key31.north) {key};

            \node[draw, minimum size=0.75cm, right=0.4375cm of key31] (key32) {};
            \node[above, yshift=-0.1cm] at (key32.north) {value};

            \node[draw, minimum size=0.75cm, right=0.4375cm of key32] (key33) {};
            \node[above, yshift=-0.1cm] at (key33.north) {next};

            \draw (key33.south west) -- (key33.north east);
            \draw (key33.north west) -- (key33.south east);

            \draw[fill=black] (key11.mid) ++(0,-0.1cm) circle (0.07);
            \draw[fill=black] (key12.mid) ++(0,-0.1cm) circle (0.07);

            \draw[fill=black] (key21.mid) ++(0,-0.1cm) circle (0.07);
            \draw[fill=black] (key22.mid) ++(0,-0.1cm) circle (0.07);

            \draw[fill=black] (key31.mid) ++(0,-0.1cm) circle (0.07);
            \draw[fill=black] (key32.mid) ++(0,-0.1cm) circle (0.07);

            \draw[->, ultra thick, line width=1.5pt] (key31.mid) ++(0,-0.1cm) -- ++(0, -1.5cm);
            \draw[->, ultra thick, line width=1.5pt] (key32.mid) ++(0,-0.1cm) -- ++(0, -1.5cm);

            \draw[->, ultra thick, line width=1.5pt] (key21.mid) ++(0,-0.1cm) -- ++(0, -1.5cm);
            \draw[->, ultra thick, line width=1.5pt] (key22.mid) ++(0,-0.1cm) -- ++(0, -1.5cm);

            \draw[->, ultra thick, line width=1.5pt] (key11.mid) ++(0,-0.1cm) -- ++(-1.2cm, 1.2cm);
            \draw[->, ultra thick, line width=1.5pt] (key12.mid) ++(0,-0.1cm) -- ++(1.2cm, 1.2cm);
        \end{tikzpicture}
        \captionsetup{position=bottom}
        \caption{Implémentation de la structure de table de hashage du module 
        hastable}\label{hastable-fig}
    \end{figure}

    \subsection{AVL}\label{avl}

    L'utilisation des arbres binaires nous a demandé la création d'un type, 
    hcell pour permettre de stocker les deux valeurs (chaine et compteur, voir 
    figure~\ref{test} pour voir une figure représentant cette structure). 
    Contrairement à la table de hachage notre implémentation des arbres binaires 
    garantie une recherche au maximum en temps logarithmique. Mais un ajout 
    dans un AVL peut entrainer au plus deux rotations de temps constant, après 
    avoir effectué son ajout en bout de chemin ce qui donne une complexité 
    logarithmique.\\
    
    Divers tests aillant pour but d'identifier la meilleure option sont effectuées 
    dans la section test.

    \section{Module de gestion d'option, optl}\label{opt}

    Le développement du module d'option est la partie du projet qui nous a été 
    le plus chronophage. En effet, nous avons dès le début eu pour objectifs de 
    produire un module générique, réutilisable pour de nombreux programmes. A 
    l'instar du module getopt qui nous semblait manquer certaines fonctionnalités
    très importante telle que la représentation des options par une chaine de 
    caractère (représentation longue), mais aussi d'autre chose présente dans 
    les options Linux. 

    Dans un premier temps, il nous a fallu effectue des recherches sur la 
    gestion d'option des programmes Linux. C'est après ces recherches que nous 
    avons remarqué avec stupeur que les programme Linux n'ont pas de norme pour 
    définir pour leur option. Nous avons donc décidé de nous en créer une en 
    nous inspirant des commandes telles que cat, ls ou encore rm du système Linux.

    Les options peuvent être représentées par deux identificateurs, une version 
    courte et une longue. Les deux ne sont pas obligatoires, mais une des deux 
    doit au moins exister. Toute option à possibilité d'interrompre le traitement 
    d'option. Toute option peut exiger un argument pour effectuer son 
    traitement. Toutes les options doivent avoir une description. 

    \subsection{Définition d'option}

    \subsubsection{Les options courtes}

    Une option courte est composée d'un `-' suivit d'un caractère 
    alphanumérique. Pour donner un paramètre à une option courte, il suffit de 
    le séparer d'un espace On peut faire appelle à plusieurs options courtes en un 
    seul  appelle, en suivant le '-` des caractères représentant les options 
    voulus. Cependant, il ne peut y avoir dans cette forme d'appel qu'un seul 
    paramètre demandant un argument et il doit alors être le dernier de la 
    liste. 

    \subsubsection{Les options longues}

    Une option longue est préfixée par la chaine `- -'. Pour donner un paramètre 
    à une longe, il faut séparer celle-ci de son argument par le caractère `='. 
    Si un utilisateur passe en paramètre une option longue qui est préfixée d'une
    et une seule option alors l'option préfixée sera appeler. Si elle est préfixe 
    de plusieurs options, alors elle sera considérer comme ambiguë, et conduira à 
    la lever d'une erreur.

    \subsubsection{Option help}

    Nous avons aussi décidé de rendre obligatoire une option, l'option `help'.
    Cette option, représenté par `-h', `--help', doit afficher une possible 
    description du programme, comment l'utiliser, mais aussi la liste de toutes 
    les options suivies de leur possible description. Cette option interrompt le 
    traitement d'options. 

    \subsection{Implémentation}

    \subsubsection{Spécification}

    Pour l'implémentation, nous avons décidé de permettre à l'utilisateur de 
    pouvoir modifier certaine chose. Notamment les préfixe des options courtes et 
    longue (On peut notamment envisager des utilisateurs originaires de Windows qui 
    préférerait utiliser le `\\' au `-'), l'identificateur des deux 
    représentation de l'option `help'. Nous avons aussi été contraints d'ajouter 
    un spécificateur permettant de garantir que la valeur suivant celui-ci n'est pas
    une option (lui aussi est modifiable par l'utilisateur du module). Sans ce 
    spécificateur, l'utilisateur ne pourrait pas rentrer la valeur `--help' en 
    prenant cette valeur non pas comme une option, mais une valeur à traiter. Ce 
    spécificateur vaux par défaut la chaine `- -'. De plus, nous avons mis en place 
    un possible traitement sur les éléments qui ne sont pas des options.
    
    \subsubsection{Le code}

    Pour regrouper toutes les informations nécessaires à la gestion d'une option 
    nous avons donc mis en place un type optparam regroupant toutes ces 
    informations. La fonction opt\_init initialise une instance de ce type. La 
    véritable difficulté a été dans la conception de la fonction de traitement,
    opt\_process. Cette fonction peut être divisée de la façon suivante :

    \newpage

    \begin{figure}[ht]
        \begin{algorithmic}
            \WHILE{argument \textbf{de} tableau-argument}
                \IF{argument représente NEXT\_NOPT}
                    \STATE{
                        \vspace{\spacebox}
                        \fcolorbox{red}{white}{
                        \begin{minipage}[c][8ex]{13cm}
                            Traitement de NEXT\_NOPT (le spécificateur que la 
                            prochaine valeur ne doit pas être considérée comme 
                            une option). 
                        \end{minipage}
                        }
                        \vspace{\spacebox}
                        }
                \ELSIF{argument est une option longue}

                    \STATE{
                        \vspace{\spacebox}
                        \fcolorbox{green}{white}{
                        \begin{minipage}[c][8ex]{13cm}
                            Traitement de l'option longue. Pour cela, un appel
                            à la fonction opt\_parse\_long visent à trouver 
                            l'option dont argument est le seul préfixe. Puis la 
                            fonction liée à l'option est exécuté. 
                        \end{minipage}
                        }
                        \vspace{\spacebox}
                        }
                \ELSIF{argument est une option courte}
                    \STATE{
                        \vspace{\spacebox}
                        \fcolorbox{black}{white}{
                        \begin{minipage}[c][8ex]{13cm}
                            Traitement des possibles options courtes. Pour ce 
                            faire, un traitement sur chaque caractère est 
                            effectué permettant de trouver toutes les 
                            options représentées dans un argument. Puis effectue le 
                            traitement lié à ces appelle.
                        \end{minipage}
                        }
                        \vspace{\spacebox}
                        }
                \ELSE{}
                    \STATE{
                        \vspace{\spacebox}
                        \fcolorbox{blue}{white}{
                        \begin{minipage}[c][8ex]{13cm}
                            Traitement de ce qui n'est pas une option, à l'aide 
                            la fonction hdl\_dlt possiblement fournit par 
                            l'utilisateur.
                        \end{minipage}
                        }
                        \vspace{\spacebox}
                        }
                \ENDIF{}
            \ENDWHILE{}
        \end{algorithmic}
        \caption{Traitement du tableau des argument par la fonction 
        opt\_process.}
    \end{figure}

    \subsection{Défaut, avantage}

    Ce module a été conçu pour être utilisé dans de nombreux cas. Il pourrait 
    donc resservi pour des futurs projets. Cette généralité entraine par contre 
    une grande difficulté quant à la compréhension de ces fonctionnalités, on 
    peut notamment citer la Spécification de la fonction otp\_process qui est 
    trop longue.
    Un point fort du reste qu'il est complet, il met en pratique toute les 
    fonctionnalités des commandes Linux les plus connus (rm, ls, cat), notamment d
    des fonctionnalités telles que `l'autocomplétion' des longues options, la prise
    en charge des multiples options courte ou encore le spécifier NEXT\_NOPT 
    (next is not an option). Un autre point fort, la complexité du traitement 
    des options, toutes les options ne sont parcourues qu'au plus deux fois pour 
    exécuter! leur traitement (avec la possible comparaison avec l'option 
    `help'). De même, ce module ne nécessite que le stockage des objets de type 
    optparam pour traiter les options.

    \section{Mise en pratique}\label{test}

    Tout au long de ce compte rendu, nous avons évoqué l'implémentation de 
    programme sans parler de la pratique. Dans cette section, nous allons voir
    quelque tests nous semblent intéressants.
    Les temps d'exécutions donnés ne sont qu'à titre indicatif et ne représentent en 
    aucun cas une promesse, étant donné que ces temps dépends de nombreux 
    facteur. En revanche l'utilisation mémoire du programme pourra être évoqué 
    dans certaine comparaison. En effet, tous les tests étant effectué sur la 
    même machine, nous supposons donc que l'utilisation mémoire du programme 
    n'évolue pas pour un même traitement. De plus, les complexités étant 
    universelle, nous favoriserons l'étude de celle-ci.

    \subsection{Test}

    \subsubsection{AVL vs Hashtable}\label{avl-has}

    Nous avons mis à disposition l'utilisation, de l'AVL ou de la Hashtable sans 
    pour autant évoquer les divers cas favorable à leur utilisation. Vous 
    retrouverait ci-dessous une étude de divers tests traitent de ces cas.

    \vphantom{}

    \SU{user=root,host=ubuntu,color=lime}
    \begin{ubuntu}
 ./lnid lesmiserables.txt `\StartConsole`
lesmiserables.txt
...
0,04s user 0,01s system 99`\%` cpu 0,051 total
`\SU{user=root,host=ubuntu,color=lime}`
./lnid lesmiserables.txt --avl `\StartConsole`
lesmiserables.txt
...
0,08s user 0,01s system 99`\%` cpu 0,093 total
`\SU{user=root,host=ubuntu,color=lime}`
./lnid HugoLeDernierJourDunCondanne.txt `\StartConsole`
HugoLeDernierJourDunCondanne.txt
...
0,00s user 0,00s system 92`\%` cpu 0,003 total
`\SU{user=root,host=ubuntu,color=lime}`
./lnid HugoLeDernierJourDunCondanne.txt --avl `\StartConsole`
HugoLeDernierJourDunCondanne.txt
...
0,00s user 0,00s system 91`\%` cpu 0,005 total
    \end{ubuntu}

    \begin{ubuntu}
`\SU{user=root,host=ubuntu,color=lime}`
./lnid liaisons_dangeureuses.txt `\StartConsole`
liaisons_dangeureuses.txt
...
0,01s user 0,00s system 96`\%` cpu 0,014 total
`\SU{user=root,host=ubuntu,color=lime}`
./lnid liaisons_dangeureuses.txt --avl `\StartConsole`
liaisons_dangeureuses.txt
...
0,01s user 0,01s system 98`\%` cpu 0,019 total
    \end{ubuntu}

    \vphantom{}

    Comment nous pouvons le voir ci-dessus, dans les divers textes de 1 à 2 Mo 
    de longueur de lignes d'environs 80 caractères. Les AVL sont légèrement plus 
    lent pour autant l'utilisation des ressources mémoire sont-elles équivalente.

    Une question se pose alors, quelle est donc l'intérêt de l'implémentation 
    des AVL\@? Les éléments présents dans une table de hachages étant 
    théoriquement accessible en temps constant. Pour cela, nous devons revenir sur 
    la mise en pratique des tables de hachage. En effet, dans notre cas, chaque 
    ligne devra calculer par la fonction de hachage. Or cette fonction (conseillé 
    par Bernstein) effectue un traitement en temps linéaire ($\Theta(n)$) sur 
    le nombre de caractères de la ligne. Au contraire pour les AVL, cette 
    recherche s'effectue par une descente de l'arbre à l'aide de la fonction 
    \textbf{strcmp}, recherche qui aura donc une complexité dans le pire des cas 
    de $O(nln*m)$ avec $n$ le nombre de lignes présente dans l'AVL et $m$ le 
    nombre de caractères de la plus grande ligne. Mais dans le meilleur des cas 
    ce traitement s'effectuera en temps constant (cette recherche est donc borné 
    par $\Omega(1)$ et $O(nln*m)$). On peut alors très vite voir que dans le cas 
    où il y a une grande diversité de ligne et qu'elles sont toutes très longues 
    l'implémentation des AVL sera plus performante que celle de la table de 
    hashage, voir figure \ref{prog-test} pour un tel exemple avec un fichier texte générer 
    à l'aide du programme de la figure \ref{prog}. 
    
    \begin{figure}[H]
        \begin{lstlisting}[style=Cstyle]
#define MAX 20000
#define LEN 80000

int main(void) {
    char alp[] = "abcdefghijklmnopqrstuvwxyzABCDEFGHIJKLMNOPQRSTUVWXYZ"
            "0123456789,;:!?./*-+@&\"'(-_)=+<> ";
    size_t alp_len = strlen(alp);
    srand((unsigned int) time(NULL));
    for (long int k = 0; k < MAX; ++k) {
        for (long int j = 0; j < LEN; ++j) {
            putchar(alp[(size_t) ((float) rand() / (float) RAND_MAX * 
                    (float) alp_len)]);
        }
        putchar('\n');
    }
    return EXIT_SUCCESS;
}\end{lstlisting}
    \captionsetup{position=bottom}
    \caption{\centering Programme C de génération de fichier text prouvant la supérioté des 
    AVL sur des fichiers à longues lignes.}\label{prog}
    \end{figure}
    \begin{figure}[H]
        \begin{ubuntu}
`\SU{user=root,host=ubuntu,color=lime}`
./lnid a.txt `\StartConsole`
a.txt
...
14,30s user 0,73s system 99`\%` cpu 15,026 total
`\SU{user=root,host=ubuntu,color=lime}`
./lnid a.txt --avl `\StartConsole`
a.txt
...
9,46s user 0,71s system 99`\%` cpu 10,169 total
        \end{ubuntu}
        \captionsetup{position=bottom}
        \caption{\centering Test du programme lnid avec l'utilisation de la table de 
        hashage et de l'AVL sur le fichier a.txt produit par le programme de la 
        figure \ref{prog}}\label{prog-test}
    \end{figure}

    Pour finir, il est nettement plus intéressant d'utiliser la version `de base'
    c'est-à-dire avec l'utilisation de la table de hachage tous simplement, car 
    dans la majorité des cas, elle sera plus performante. Tout de même dans le 
    cas où l'utilisateur connait la taille moyenne de ces lignes, il lui pourrait 
    être utile d'utiliser alors l'AVL\@.

    \subsubsection{Le cout du traitement d'options}\label{test-cmp}

    On peut classer les traitements d'options du programme en plusieurs catégories 
    (avl exclu aillant déjà était traité longuement dans la partie \ref{avl-has}
    ):
    
    \begin{enumerate}
        \item[] \textbf{modification affichage erreur},  avec l'option 
        \textbf{no-color}. Cette option à un traitement exécutable en 
        complexité constante.
        \item[] \textbf{l'affichage d'information}, les options \textbf{help} et 
        \textbf{version}. Leur traitement étant en complexité constante, mais 
        aussi leur spécificité à stopper le logiciel, il ne constitue alors 
        qu'une complexité globale de $O(n)$ avec $n$ le nombre d'options à 
        traiter. Cette complexité résulte du possible traitement des n options 
        possiblement placer avant l'une de ces deux options.
        \item[] \textbf{trie}, de l'option \textbf{sort}. Ce trie s'effectuant 
        sur le `four-tout', il ne change en rien la lecture des lignes. Ce trie 
        est un tri par fusion, donc en complexité $\Theta(nln)$ avec $n$ le 
        nombre de lignes (toujours en supposant que les fonctions de comparaison 
        \textbf{strcmp} et \textbf{strcoll} (avec respectivement la version 
        \textbf{standard} et \textbf{local} de l'option \textbf{sort}) 
        s'exécute en temps constant), ce qui veut dire qu'il suffit d'ajouter 
        cette complexité à celle du programme, complexité qui sera donc 
        négligeable en théorie. Or dans la pratique, on remarque avec les tests de
         la figure \ref{test-cmp-f} que les fonctions strcoll et strcmp ne sont 
        pas constante, mais qu'une certaine différence semble apparaitre. En
        effet, la fonction \textbf{strcmp} est généralement plus rapide que 
        \textbf{strcoll}. Cela est dû au fait que strcmp compare les chaînes de 
        caractères en utilisant simplement les caractères par leurs 
        représentation, tandis que strcoll prend en compte les différences 
        culturelles dans l'ordre des caractères et utilise des règles de tri 
        spécifiques à l'ordre local.        
        \item[] \textbf{modification}, l'option \textbf{uppercasing}. Ce 
        traitement présent dans la fonction de lecture de ligne \textbf{fnlines} 
        oblige la modification de tous les caractères lus. Même dans le cas ou 
        cette option n'est pas appelée par l'utilisateur, il force un teste 
        (pour savoir si une telle modification est nécessaire). Donc ce 
        traitement n'ajoute qu'une action en cas d'appel à cette option (en 
        supposant que la fonction \textbf{toupper} s'exécute en temps constant).
        \item[] \textbf{filtrage}, l'option \textbf{filter}. Ce traitement tous 
        comme la modification, ajout une action dans le pire des cas, qui 
        correspond à celui ou l'option est choisie par l'utilisateur (en 
        supposant que les fonctions is... s'exécuté en temps constant). 
    \end{enumerate}

    Ce qui revient toujours à un traitement de $O(n)$ pour toute ligne lue.

    \begin{figure}[H]
        \SU{user=root,host=ubuntu,color=lime}
        \begin{ubuntu}
./lnid lesmiserables.txt -s local `\StartConsole`
lesmiserables.txt
44941,45572	_16, rue de la Verrerie_.
9074,10699,35854,59555	--Ah!
...            
8982,24187	--Vous?
26484,36185	X.
0,10s user 0,01s system 97`\%` cpu 0,115 total
`\SU{user=root,host=ubuntu,color=lime}`
./lnid lesmiserables.txt -s standard `\StartConsole`
lesmiserables.txt
35804,35861     _S`\textquotesingle`en allait a la chasse,_
47234,47258     _Demandait Charlot a Charlotte._
...
44552,47345	tout le monde.
20628,35424,40684,61779	vous.
0,07s user 0,00s system 99`\%` cpu 0,061 total
        \end{ubuntu}
        \captionsetup{position=bottom}
        \caption{\centering Comparaison entre strcmp et strcoll, sur le fichier 
            text \textit{lesmiserables.txt}. Où \textbf{local}, 
            \textbf{standard} correspond à l'utilisation de \textbf{strcoll} et 
            \textbf{strcmp}.}\label{test-cmp-f}
    \end{figure}

    \newpage

    \subsection{Performance}

    Pour cette section, nous allons présenter plusieurs tests que nous jugeons 
    Intéressants pour juger les performances du programme. Le 
    premier test est effectué sur un fichier de 3Mo contenant des lignes de 
    longueur d'au plus 80 caractères (les misérables de Victor Hugo, ce test 
    n'est pas effectué avec l'option \textbf{AVL}, voir section~\ref{avl} pour 
    la justification de ce choix). On observe sur la figure~\ref{our-prog}, un 
    temps d'exécution de l'ordre de la centaine de millisecondes et une 
    utilisation mémoire de 9 millions de bytes, ce qui revient à une 
    multiplication d'environ 2,97 fois la taille du fichier.

    \begin{figure}[H]
        \SU{user=root,host=ubuntu,color=lime}
        \begin{ubuntu}
./lnid lesmiserables.txt `\StartConsole`
lesmiserables.txt
...
0,07s user 0,00s system 99`\%` cpu 0,071 total
`\SU{user=root,host=ubuntu,color=lime}`
valgrind ./lnid lesmiserables.txt `\StartConsole`
...
heap usage: 299,780 allocs, 299,780 frees, 8,925,910 bytes allocated
        \end{ubuntu}
        \captionsetup{position=bottom}
        \caption{Démonstration d'exécution de l'exécutable lnid.}\label{our-prog}
    \end{figure}

    Le deuxième test est effectué sur un fichier de 4 Go contenant des lignes de
    longueurs de 800 000 caractères à l'aide de l'option AVL (ce fichier ne 
    contient aucune ligne identique).\@On observe sur la figure~\ref{avl-4go} un 
    temps d'exécution de l'ordre de vingtaine de seconde et une utilisation 
    mémoire de 4 milliards de bytes ce qui revient à une multiplication 
    d'environ 1,001 fois la taille du fichier.

    \begin{figure}[H]
        \SU{user=root,host=ubuntu,color=lime}
        \begin{ubuntu}
./lnid a.txt `\StartConsole`
a.txt
...
21,97s user 1,50s system 100`\%` cpu 23,462 total
`\SU{user=root,host=ubuntu,color=lime}`
valgrind ./lnid a.txt `\StartConsole`
...
heap usage: 35,036 allocs, 35,036 frees, 4,002,711,191 bytes allocated
        \end{ubuntu}
        \captionsetup{position=bottom}
        \caption{Démonstration d'exécution de l'exécutable de lnid.}\label{avl-4go}
    \end{figure}

    On remarque alors que sur les deux exemples présentés

    \newpage

    \section{Conclusion}

    Pour finir, la production de ce programme aura été de notre côté linéaire. 
    Lors de son développement, nous n'avons pas ressenti de réelles difficultés. 
    Cependant, notre envie de toujours faire mieux nous à poser un problème. On 
    peut notamment citer les cinq versions du module opt que nous avons conçue, 
    car nous n'étions jamais satisfaits. De plus, une autre difficulté 
    rencontrée a été la production de ce compte rendu. N'aillant jamais 
    effectuer d'exercice de ce type, nous avons donc essayé de présenter notre 
    implémentation de notre programme. Malgré tout se travaille, nous avons 
    déjà en tête quelque possible amélioration, un affichage du help du module 
    optl qui pourrait se s'adapter à la taille du terminal. Toujours dans le 
    module optl, donner des options avec un paramètre qui n'est pas forcément 
    obligatoire (qui ne déclenche donc pas d'erreur s'il n'est pas donnée). Ou 
    encore pour le programme \textbf{lnid}, il serait peu intéressant de 
    s'intéresser aux lignes identiques au sens de la phrase et non identique en 
    sens du contenu.

    \subsection{Remerciment}

    Merci à\@: 

    Erwan Lievin, pour le nom qu'il nous a fourni pour le module de tableau 
    dynamique `da' (dynamique array).

    Au logiciel ChatGPT pour la correction des nombreuses fautes d'orthographe 
    présent tout au long de la production de ce projet et de son compte rendu.

    Ilyas TAKHTOUKH, pour avoir effectué une relecture des fautes 
    d'orthographes présentes dans ce compte-rendu est dans les spécifications du 
    code source du programme. 

    \newpage

    \section{Bibliographie}
        \noindent Concernant la partie du module d'option optl 
        (section \ref{opt}):\\ Présentation des divers `tradition' de gestion 
        d'option pour UNIX et GNU, 
        \url{http://www.catb.org/~esr/writings/taoup/html/ch10s05.html}.\\
        Description de la gestion des options en ligne à la GNU, 
        \url{https://www.gnu.org/software/gawk/manual/html_node/Options.html}.\\
        Documentation du module getopt, pour possiblement faire une comparaison 
        entre ce module et optl, 
        \url{https://www.gnu.org/software/libc/manual/html_node/Getopt.html}.\\
        \noindent Concernant la partie Test (sous-section \ref{test-cmp}):\\
        Comparaison entre la fonction \textbf{strcmp} et \textbf{strcoll}, 
        \url{https://learn.microsoft.com/en-us/cpp/c-runtime-library/strcoll-functions?view=msvc-170}.
\end{document}